\documentclass[a4paper,12pt]{amsart}
\usepackage[utf8]{inputenc}\usepackage[T1]{fontenc}

\title[Deep Deterministic Policy Gradients]{The Tennis environment solved using a Multi-Agent Deep Deterministic Policy Gradient algorithm}
\author{Julian Wergieluk}\address{}\email{julian.wergieluk@risklab.com}
%%% A generic preamble. 

\usepackage[english]{babel} 


\usepackage[final]{graphicx} % options = [final]  = all graphics displayed, regardless of draft option in class
                               % options = [pdftex] = necessary (?) if import PDF files
                               % no option : when importing ps- and eps-files (?)
% \graphicspath{{../images/}}  % tell LaTeX where to look for images
% \DeclareGraphicsExtensions{.pdf, .PDF, .jpg, .JPG, .jpeg, .JPEG, .png, .PNG, .bmp, .BMP, .eps, .ps}
\usepackage{float}                      % Improved interface for floating objects ; add [H] option

\usepackage[usenames,dvipsnames,svgnames,table]{xcolor}
%\usepackage{listings}%\lstset{basicstyle=\ttfamily, breaklines=true}
\usepackage{verbatim}
%\makeindex

% memoir options
%\chapterstyle{companion}

% ams packages and symbols
\usepackage{amsmath}      % loads amstext, amsbsy, amsopn but not amssymb
                            % equation stuff (eqref, subequations, equation, align, gather, flalign, multline, alignat, split...)
\usepackage{amsfonts}     % may be redundant with amsmath
% \usepackage{amssymb}      % may be redundant with amsmath
\usepackage{latexsym}

% layout
\usepackage[final, babel]{microtype} % many good lay-out/justification effects, see:
                                     % texblog.net/latex-archive/layout/pdflatex-microtype/

\usepackage{mathtools}%                  http://www.ctan.org/pkg/mathtools
\DeclarePairedDelimiter\abs{\lvert}{\rvert}

\usepackage[tableposition=top]{caption}% http://www.ctan.org/pkg/caption
\captionsetup[table]{skip=10pt}
%\usepackage{booktabs,dcolumn}%           http://www.ctan.org/pkg/dcolumn + http://www.ctan.org/pkg/booktabs
%\usepackage{textpos}     % insert a text at an absolute position

\usepackage{amsthm, thmtools}

\theoremstyle{plain}
\newtheorem{theorem}{Theorem}[section]
\newtheorem{acknowledgement}[theorem]{Acknowledgement}
\newtheorem{algorithm}[theorem]{Algorithm}
\newtheorem{axiom}[theorem]{Axiom}
\newtheorem{case}[theorem]{Case}
\newtheorem{claim}[theorem]{Claim}
\newtheorem{conclusion}[theorem]{Conclusion}
\newtheorem{conjecture}[theorem]{Conjecture}
\newtheorem{corollary}[theorem]{Corollary}
\newtheorem{criterion}[theorem]{Criterion}
\newtheorem{exercise}[theorem]{Exercise}
\newtheorem{lemma}[theorem]{Lemma}
\newtheorem{notation}[theorem]{Notation}
\newtheorem{problem}[theorem]{Problem}
\newtheorem{proposition}[theorem]{Proposition}
\newtheorem{solution}[theorem]{Solution}
\newtheorem{definition}[theorem]{Definition}
\newtheorem{condition}[theorem]{Condition}
\newtheorem{assumption}[theorem]{Assumption}
\newtheorem{model}[theorem]{Model}

\theoremstyle{remark}
\newtheorem{remark}[theorem]{Remark}
%\newtheorem{example}[theorem]{Example}
\newtheorem{summary}[theorem]{Summary}
\newtheorem{questions}[theorem]{Questions}

%%% FONTS

%\usepackage{concmath}
\usepackage{fourier}
\usepackage{newcent}


% HYPERREF (last) then HYPCAP %%%%%%%%%%%%%%%%%%%%%%%%%%%%%%%%%%%%%%%%%%%%%%%%%

%
% See:
% http://tex.stackexchange.com/questions/1863/which-packages-should-be-loaded-after-hyperref-instead-of-before
%
\usepackage[
pdftex, 
final,                      % if you do    want to have clickable-colorful links
pdfstartview = FitV,
linktocpage  = false,       % ToC, LoF, LoT place hyperlink on page number, rather than entry text
breaklinks   = true,        % so long urls are correctly broken across lines
% pagebackref  = false,     % add page number in bibliography and link to position in document where cited
]{hyperref}
\hypersetup{
    colorlinks=true, 
    linkcolor={black},
    citecolor={black},
    urlcolor={black},
    bookmarksopenlevel={2}, 
    bookmarksopen = true
}

% \usepackage{cleveref} % enhance cross-referencing, allow full formatting, commands:
                        % \cref, \Cref, \crefrange, \cref{eq2,eq1,eq3,eq5,thm2,def3}
                        % supposedly better than \autoref as provided by hyperref

% \usepackage[all]{hypcap} % when link to float (using hyperref), link anchors to beginning (instead of below) float




% a bar that is slightly shorter than \overline
\newcommand{\overbar}[1]{\mkern 1.5mu\overline{\mkern-1.5mu#1\mkern-1.5mu}\mkern 1.5mu}

% Annotation
\newcommand{\key}[1]{\emph{#1}\index{#1}}
\newcommand{\BUL}{\noindent\quad\textbullet\quad }

% General
\newcommand{\R}{\mathbb{R}}
\newcommand{\eR}{\overbar{\R}}  % extended reals
\newcommand{\N}{\mathbb{N}}
\newcommand{\ind}{1}

%Stochastics
\newcommand{\E}{\mathbb{E}}
\newcommand{\cond}{\;\middle\vert\;}
\newcommand{\vb}{\;|\;}
\newcommand{\condE}[2]{\E\left[ #1 \,|\, #2 \right] }
\newcommand{\condP}[2]{\bP\left[ #1 \,|\, #2 \right] }
\newcommand\upmodels{\protect\mathpalette{\protect\independenT}{\perp}}
\def\independenT#1#2{\mathrel{\rlap{$#1#2$}\mkern2mu{#1#2}}}
\DeclareMathOperator*{\esssup}{ess\,sup}
\DeclareMathOperator*{\essinf}{ess\,inf}
\newcommand{\sint}{\bullet} %% operator notation for stochastic integral
\newcommand{\dchar}[1]{\partial{#1}} %% differential characteristics of a semimartingale

% Logic
\newcommand{\setcolon}{:}
\newcommand{\impl}{\Rightarrow}
\DeclareMathOperator{\cov}{cov}
\DeclareMathOperator{\corr}{corr}
\DeclareMathOperator{\var}{var}\DeclareMathOperator{\Var}{Var}

\renewcommand{\div}{\mathop\backslash}

\newcommand{\bA}{\mathbb A}\newcommand{\cA}{\mathcal A}\newcommand{\fA}{\mathfrak A}\newcommand{\bfA}{\mathbf A}
\newcommand{\bB}{\mathbb B}\newcommand{\cB}{\mathcal B}\newcommand{\fB}{\mathfrak B}\newcommand{\bfB}{\mathbf B}
\newcommand{\bC}{\mathbb C}\newcommand{\cC}{\mathcal C}\newcommand{\fC}{\mathfrak C}\newcommand{\bfC}{\mathbf C}
\newcommand{\bD}{\mathbb D}\newcommand{\cD}{\mathcal D}\newcommand{\fD}{\mathfrak D}\newcommand{\bfD}{\mathbf D}
\newcommand{\bE}{\mathbb E}\newcommand{\cE}{\mathcal E}\newcommand{\fE}{\mathfrak E}\newcommand{\bfE}{\mathbf E}
\newcommand{\bF}{\mathbb F}\newcommand{\cF}{\mathcal F}\newcommand{\fF}{\mathfrak F}\newcommand{\bfF}{\mathbf F}
\newcommand{\bG}{\mathbb G}\newcommand{\cG}{\mathcal G}\newcommand{\fG}{\mathfrak G}\newcommand{\bfG}{\mathbf G}
\newcommand{\bH}{\mathbb H}\newcommand{\cH}{\mathcal H}\newcommand{\fH}{\mathfrak H}\newcommand{\bfH}{\mathbf H}
\newcommand{\bI}{\mathbb I}\newcommand{\cI}{\mathcal I}\newcommand{\fI}{\mathfrak I}\newcommand{\bfI}{\mathbf I}
\newcommand{\bJ}{\mathbb J}\newcommand{\cJ}{\mathcal J}\newcommand{\fJ}{\mathfrak J}\newcommand{\bfJ}{\mathbf J}
\newcommand{\bK}{\mathbb K}\newcommand{\cK}{\mathcal K}\newcommand{\fK}{\mathfrak K}\newcommand{\bfK}{\mathbf K}
\newcommand{\bL}{\mathbb L}\newcommand{\cL}{\mathcal L}\newcommand{\fL}{\mathfrak L}\newcommand{\bfL}{\mathbf L}
\newcommand{\bM}{\mathbb M}\newcommand{\cM}{\mathcal M}\newcommand{\fM}{\mathfrak M}\newcommand{\bfM}{\mathbf M}
\newcommand{\bN}{\mathbb N}\newcommand{\cN}{\mathcal N}\newcommand{\fN}{\mathfrak N}\newcommand{\bfN}{\mathbf N}
\newcommand{\bO}{\mathbb O}\newcommand{\cO}{\mathcal O}\newcommand{\fO}{\mathfrak O}\newcommand{\bfO}{\mathbf O}
\newcommand{\bP}{\mathbb P}\newcommand{\cP}{\mathcal P}\newcommand{\fP}{\mathfrak P}\newcommand{\bfP}{\mathbf P}
\newcommand{\bQ}{\mathbb Q}\newcommand{\cQ}{\mathcal Q}\newcommand{\fQ}{\mathfrak Q}\newcommand{\bfQ}{\mathbf Q}
\newcommand{\bR}{\mathbb R}\newcommand{\cR}{\mathcal R}\newcommand{\fR}{\mathfrak R}\newcommand{\bfR}{\mathbf R}
\newcommand{\bS}{\mathbb S}\newcommand{\cS}{\mathcal S}\newcommand{\fS}{\mathfrak S}\newcommand{\bfS}{\mathbf S}
\newcommand{\bT}{\mathbb T}\newcommand{\cT}{\mathcal T}\newcommand{\fT}{\mathfrak T}\newcommand{\bfT}{\mathbf T}
\newcommand{\bU}{\mathbb U}\newcommand{\cU}{\mathcal U}\newcommand{\fU}{\mathfrak U}\newcommand{\bfU}{\mathbf U}
\newcommand{\bW}{\mathbb W}\newcommand{\cW}{\mathcal W}\newcommand{\fW}{\mathfrak W}\newcommand{\bfW}{\mathbf W}
\newcommand{\bV}{\mathbb V}\newcommand{\cV}{\mathcal V}\newcommand{\fV}{\mathfrak V}\newcommand{\bfV}{\mathbf V}
\newcommand{\bX}{\mathbb X}\newcommand{\cX}{\mathcal X}\newcommand{\fX}{\mathfrak X}\newcommand{\bfX}{\mathbf X}
\newcommand{\bY}{\mathbb Y}\newcommand{\cY}{\mathcal Y}\newcommand{\fY}{\mathfrak Y}\newcommand{\bfY}{\mathbf Y}
\newcommand{\bZ}{\mathbb Z}\newcommand{\cZ}{\mathcal Z}\newcommand{\fZ}{\mathfrak Z}\newcommand{\bfZ}{\mathbf Z}




\usepackage[url=false,backend=biber]{biblatex}
\addbibresource{rl.bib}
\addbibresource{ddpg.bib}

\newcommand{\stateSpace}{\mathbb S}
\newcommand{\stateSpaceAlg}{\mathcal S}
\newcommand{\coffinSpace}{\Delta}
\newcommand{\actionSpace}{\mathbb A}
\newcommand{\actionSpaceAlg}{\mathcal A}
\newcommand{\stateValueFunc}{V}
\newcommand{\actionValueFunc}{Q}
\newcommand{\advantageFunc}{A}
\newcommand{\policy}{\pi}
\newcommand{\policyLik}{f}
\newcommand{\discountFactor}{\gamma}
\newcommand{\prob}{\mathbb P}
\newcommand{\rewardFunc}{\phi}
\newcommand{\trajectory}{\tau}
\newcommand{\trajectorySpace}{\mathbb T}
\newcommand{\startStateDist}{\rho_0}

\begin{document}

\maketitle

\begin{abstract}
This short note provides a concise description of the model architecture and
learning algorithms of the agent developed in this project. We also report learning
performance of the agent and provide a list of possible future model improvements.
\end{abstract}

%\begin{abstract}
%This short note provides a concise description of the model architecture and
%learning algorithms of the agent developed in this project. We also report learning
%performance of the agent and provide a list of possible future model improvements.
%\end{abstract}
%\renewcommand*{\thefootnote}{}\footnote{\today{}}

\section{Description of the learning algorithm}

\subsection{Summary of the notation}

%In this write-up I am going to cast the core ideas of policy gradient methods
%into a rigorous probabilistic framework. Much of the confusion and
%difficulty of understanding reinforcement learning comes from sloppy and
%incomplete mathematical notation.

Let's setup up the stage for the show and define a probability space
$\left( \Omega, \mathcal F, \prob \right)$.
Consider a Markov Decision Process (MDP) with measurable state and action spaces
$(\stateSpace, \stateSpaceAlg)$ and $(\actionSpace, \actionSpaceAlg)$.
A policy $\policy$ is a Markov kernel from $(\stateSpace, \stateSpaceAlg)$ to
$(\actionSpace, \actionSpaceAlg)$, i.e.\ for each $s\in\stateSpace$, 
$\policy(. \vb s)$ is a probability distribution on $\actionSpaceAlg$.
A trajectory
\begin{align*}
    \trajectory = \left( S_0, A_0, R_1, S_1, A_1, R_2, \ldots \right) 
\end{align*}
encodes a sequence of states, actions and resulting
numerical rewards pertaining to a policy $\policy$. The elements of $\tau$
are random variables, such that $S_{0}$ has the \key{start-state distribution} 
$\startStateDist$ (which does not depend on the policy $\pi$), 
$\policy(. | s)$ is the conditional distribution of $A_i$ given $S_i = s$ for
any $s\in\stateSpace$, and $\prob\left(. | S_{i-1}= s, A_{i-1}=a \right)$ is
the conditional distribution of $S_{i}$. The rewards $R_i$ are random
variables taking values in $\R$. The return or discounted future reward $G_t$
at $i\in\N$ is defined as
\begin{align}
    G_i &= \sum_{k=0}^{\infty} \discountFactor^{k} R_{i+k+1},
    \label{def:return}
\end{align}
where $\discountFactor$ is a hyperparameter from the interval $(0,1)$, used to guarantee
the convergence of the series (\ref{def:return}).

Since $\trajectory$ is a sequence of random variables, and therefore itself a random
variable taking values in the trajectory space 
\begin{align*}
    \trajectorySpace = \stateSpace\times\actionSpace\times 
        \left( \stateSpace\times\actionSpace\times\R \right)^{\infty},
\end{align*}
we can sample from $\trajectory$ to obtain sequences of the form
\begin{align*}
    \left( s_0, a_0, r_1, s_1, a_1, r_2,\ldots \right)\in\trajectorySpace.
\end{align*}
These samples of $\tau$ are called \key{episodes} or \key{rollouts}.

The central object of interest is the mechanics of the world encoded in the
transition probability measure $\prob(. \vb s, a)$. For $i>0$, it is the distribution of
$S_i$ under the conditions $S_{i-1} = s$ and $A_{i-1} = a$ for fixed
$s\in\stateSpace$ and $a\in\actionSpace$. This distribution does not depend on $i$, and,
more importantly, does not depend on the policy $\policy$. 

To simplify the setup, we assume that the reward $R_i$ at time $i$ is a deterministic
function of the relevant states and actions. Depending on the problem at hand
these are the three common choices encountered in the literature:
\begin{align*}
    R_{i} &= \rewardFunc_{1}(S_{i}), \\
    R_{i} &= \rewardFunc_{2}(A_{i-1}, S_{i}), \\
    R_{i} &= \rewardFunc_{3}(S_{i-1}, A_{i-1}, S_{i})
\end{align*}
$\rewardFunc_{1}$ is probably the most frequently used form.

In reinforcement learning, we parametrize the policy $\policy_\theta$ with
an $\R^{d}$ vector $\theta$, $d\geq 1$. The goal is to find good values for $\theta$ 
which lead to high cumulative expected reward
\begin{align*}
    J\left( \theta \right) &= \E_{\theta} \left[ G_0 \right] 
    = \E_{\theta} \left[ \sum_{k=0}^{\infty} \discountFactor^{k} R_{k+1} \right].
\end{align*}
The subscript $\theta$ next to the expectation operator indicates that
the underlying probability measure $\prob_{\theta}$ combines both
$\prob$ and $\policy_{\theta}$.

\subsection{Value functions}

The \key{state-value function} is defined for each state $s\in\stateSpace$ as
\begin{align}
    \stateValueFunc_{\theta}(s) &= \E_{\theta} \left[ G_0 | S_{0} = s \right] =
    \E_{\theta}\left[\sum_{k=0}^{\infty} \discountFactor^{k} R_{k+1} | 
        S_{0} = s \right].
    \label{def:state-value-function}
\end{align}
The \key{action-value function} is defined for each state-action pair
$(s,a)\in\stateSpace\times\actionSpace$ as
\begin{align}
    \actionValueFunc_{\theta}(s,a) &= \E_{\theta} \left[ G_0 | S_0 = s, A_{0} = a \right] = 
    \E_{\theta}\left[\sum_{k=0}^{\infty} \discountFactor^{k} R_{k+1} | 
        S_0 = s, A_{0} = a \right].
    \label{def:action-value-function}
\end{align}
The \key{advantage function} is the difference
\begin{align}
    \advantageFunc_{\theta}(s,a) &= V_{\theta}(s) - Q_{\theta}(s,a).
    \label{def:advantage-function}
\end{align}

\subsection{Episodic and perpetual tasks} 
Reinforcement learning tasks can be partitioned into two categories, depending
on whether the task can be completed. A task is called \key{episodic} if
there is an end of the task, i.e.\ some notion of ``being done''.
If there is no such notion, the task is called \key{perpetual}
\footnote{In the literature, people use the adjective ``continuous''.
    But we already have ``continuous spaces'', ``continuous time'', ``continuous function'', etc.
    The word ``perpetual'' is much better at conveying the property 
    of running indefinitely long. 
}.
Typical examples of episodic tasks are games. Most board games like chess and
go, and computer games fall into this category.

It is important to note, that many episodic tasks can also run 
forever. For example, in chess, we can design a policy that ``runs in circles''
and does not pursue the goal of winning the game. Letting such a 
policy play against itself will result in an endless game.

What distinguishes an episodic task from a perpetual one, is the existence of
a non-empty set of end states $\coffinSpace$ in $\stateSpace$, aptly called the
\key{coffin space}, for which we require that $\prob\left( S_{i+1} = S_{i} |
S_{i} \in \coffinSpace \right) = 1$ and $\prob\left( R_{i} = 0 | S_{i} \in
\coffinSpace \right) = 1$. This ensures that all the states in $\coffinSpace$
are absorbing and that no additional reward can be accumulated after the task
has been completed.

\subsection{Categorial and continuous action spaces}
The cardinality of the action space has a huge impact on the development of the
theory of policy gradients as well as on the design and implementation of
the policy optimization algorithms.
Essentially, there are two cases we need to deal with. First, if the action
space has finite number of elements, we call it \key{categorial}. Finite
number of possible actions implies that the policy $\policy(.|s)$ evaluated at
the state $s$, can be identified with a vector with $|\actionSpace|$ number of
elements. Also, it is convenient to define the likelihood function $\policyLik$
of the policy $\policy$ as 
\begin{align*}
    \policyLik(a | s) = \policy(\left\{a \right\}|s), \, a\in \actionSpace.
\end{align*}
$\policyLik(.|s)$ is the probability function of the probability measure 
$\policy(.|s)$.

The second case encountered in the literature is the \key{continuous} action
space $\actionSpace$ which is a ``nice'' subset of $\R^{n_{\actionSpace}}$.
Typical example of a continuous action space is a product of finite or infinite
intervals in $\R^{n_{\actionSpace}}$. Let us assume that, for each state $s$, 
the measure $\policy(.|s)$ has a density $\policyLik(.|s)$, i.e.\ 
for an action set $\eta\in\actionSpaceAlg$ we have
\begin{align*}
    \policy(\eta | s) = \int_{\eta} \policyLik(a|s) da.
\end{align*}
As in the categorial case, we call $\policyLik(.|s)$ the likelihood function of the
measure $\policy(.|s)$.

%\subsection{Technical assumptions}
%In the following list, we gather the technical assumptions.
%\begin{assumption}
%    \begin{enumerate}
%        \item The reward function $\rewardFunc$ is measurable and bounded.
%    \end{enumerate}
%
%    \label{assumptions}
%\end{assumption}
%
%Let us state that our body of thought compiles without errors.
%\begin{theorem}
%    Suppose the conditions stated in Assumption \ref{assumptions} hold. Then we have:
%    \begin{enumerate}
%        \item The value functions $\stateValueFunc_{\theta}$, 
%            $\actionValueFunc_{\theta}$, and $\advantageFunc_{\theta}$ are well-defined.
%    \end{enumerate}
%\end{theorem}

%\subsection{Cumulative reward optimization}
%One of our goals is to find the values of the parameter $\theta$ yielding a
%policy that consistently produces high cumulative expected award $J(\theta)$.
%
%The gradient of $J(\theta)$ is given by
%\begin{align*}
%    \nabla_{\theta} J(\theta) &= 
%    \nabla_{\theta} \E_{\theta} \left[ G_{0} \right] = 
%    \E_{\theta} \left[ G_0 \sum_{i=0}^{\infty} 
%    \nabla_{\theta}\log\policy_{\theta}(A_i | S_i) \right].
%\end{align*}
%
%Using the Markov property we get
%\begin{align*}
%    \nabla_{\theta} J(\theta) &= 
%    \E_{\theta} \left[ \sum_{i=0}^{\infty} G_{i}
%    \nabla_{\theta}\log\policy_{\theta}(A_i | S_i)  \right].
%\end{align*}
%
%Policy gradient with the state-value baseline
%\begin{align*}
%    \nabla_{\theta} J(\theta) &= 
%    \E_{\theta} \left[ \sum_{i=0}^{\infty} \left( G_{i}-\stateValueFunc_{\theta}(S_i) \right)
%    \nabla_{\theta}\log\policy_{\theta}(A_i | S_i) \right].
%\end{align*}
%
%Policy gradient with the advantage function baseline
%\begin{align*}
%    \nabla_{\theta} J(\theta) &= 
%    \E_{\theta} \left[ \sum_{i=0}^{\infty} \advantageFunc_{\theta}(S_i, A_i)
%    \nabla_{\theta}\log\policy_{\theta}(A_i | S_i) \right].
%\end{align*}

\subsection{Deep deterministic policy gradients}

Deep deterministic policy gradient algorithm is an actor-critic type method
with a deterministic actor policy $\pi$ mapping states from $\stateSpace$ to
actions in $\actionSpace$. During the training, the action-value
function $Q$ is optimized using a standard DQN algorithm as described in
\cite{mnih2015humanlevel} employing a replay buffer and a separate target
network.

The policy network is trained using a policy gradient derived from the action-value
function approximation. Specifically, the parameter vector $\theta$ is given as
$\theta = (\theta^{Q}, \theta^{\pi})$, where $\theta^{Q}$ determines $Q$ and
$\theta^{\pi}$ determines $\pi$. Let $J(\theta)$ be the cumulative expected 
reward. Then the policy gradient can be approximated by
\begin{align*}
    \nabla_{\theta^{\pi}} J & \approx
    \E \left[ \nabla_{a} Q(S_t, \pi(S_t)) \nabla_{\theta^\pi} \pi(S_t) \right].
\end{align*}

To improve the algorithm stability we use soft target updates for the
parameters of both actor and critic networks, described in
\cite{mnih2015humanlevel}, and batch normalization.

\section{Training analysis}

We train the DDPG agent for 250 episodes consisting of 1000 time steps each. The cumulative
reward averaged over 100 episodes reaches the level of 30.23 after completing the episode
124. 

\begin{figure}[tb]
    \centering
    \includegraphics[width=\textwidth]{{scores}.pdf}
    \caption{We train a DDPG agent for 250 episodes. The ``reward'' curve gives
        the average reward for all 20 arms at the end of each episode. The
        ``average reward'' curve is the rolling mean of the ``reward'' curve
        calculated over 100 episodes. (For episodes $1,\cdots, 99$ we use an
        expanding window.)}
    \label{fig:scores}
\end{figure}

\section{Algorithm parametrization}

\paragraph{The Actor network architecture.} The actor is a feed-forward neural
network mapping the state of the environment to a deterministic action. The
network consists of three layers sandwiched with batch normalization. The final
layer uses the tanh nonlinearity to produce an action vector in the cube $[-1, 1]^{4}$.

\verbatiminput{actor-net.txt}

\paragraph{The Critic network architecture.} The critic is a feed-forward neural 
network approximating the state-action value function $Q$ mapping state-action
pairs to the expected cumulative reward. The critic network consists of two subnetworks: 
a one-layer network combined with batch normalization that maps the state to a high-dimensional 
state representation, and a two-layer network that maps the aforementioned representation
of the state concatenated with the action vector to the approximation of the state-action
value $Q(s,a)$.

\verbatiminput{critic-net.txt}

The list of hyperparameters used by the agent is listed in Table \ref{tab:hyperparameters}.

\begin{table}
%\centering
\caption{List of hyperparameters and their values}
\begin{tabular}{|l|l|l|l|}
    \hline
Hyperparameter & Variable name & Value \\ 
    \hline \hline
    Replay buffer size & \texttt{BUFFER\_SIZE} &  1e6 \\ \hline
    Batch size & \texttt{BATCH\_SIZE} &  128  \\  \hline
    Discount factor ($\gamma$) & \texttt{GAMMA} &  0.99 \\ \hline
    Tau & \texttt{TAU} & 1e-3 \\ \hline
    Actor learning rate & \texttt{LR\_ACTOR} & 0.001 \\ \hline
    Critic learning rate & \texttt{LR\_CRITIC} & 0.001 \\ \hline
    Initial value of the & \texttt{EPSILON} & 1.0 \\
    OU process scaling factor $\varepsilon$ && \\ \hline
    OU mean reversion level & & 0.0 \\ \hline
    OU mean reversion speed & & 0.15 \\ \hline
    OU volatility (sigma) & & 0.2 \\ \hline
    Learning frequency & \texttt{} &  every 20 time steps \\ \hline
\end{tabular}
\label{tab:hyperparameters}
\end{table}


\section{Ideas for future work}



\begin{itemize}
    \item Investigation whether the OU process generating the exploratory
        behavior of the actor could be replaced with a simple Gaussian noise.
    \item Experimenting with bigger networks.
    \item Actor and critic could use a common network for state processing. 
    \item Performance comparison with other continuous control policy gradient algorithms,
        like, e.g., PPO.
\end{itemize}


\nocite{lillicrap2015continuous}
\printbibliography

\end{document}




% vim: spelllang=en_us:spell:
